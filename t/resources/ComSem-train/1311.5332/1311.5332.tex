\documentclass{amsart}
\usepackage{amssymb}
\usepackage{xcolor}
%%% debugging
\let\oldlabel\label
%\renewcommand{\label}[1]{\oldlabel{#1}\textsf{(#1)}}
%%%
\title{On a Conjecture of Erd\H os, Gallai, and Tuza}
\author{Gregory J.~Puleo}
\renewcommand{\subset}{\subseteq}
\newcommand{\iso}{\cong}
\newcommand{\lfrac}[2]{#1/#2}
\newcommand{\reals}{\mathbb{R}}
\newcommand{\finishlater}{\bad{\bf FINISH LATER}}
\newcommand{\bad}[1]{\textcolor{red}{\emph{#1}}}
\newcommand{\citethis}[1]{\bad{[CITE THIS]}}
\newcommand{\caze}[2]{\textbf{Case {#1}:} \textit{#2}}
\newcommand{\sizeof}[1]{\left\lvert{#1}\right\rvert}
\newcommand{\floor}[1]{\left\lfloor{#1}\right\rfloor}
\newcommand{\ceil}[1]{\left\lceil{#1}\right\rceil}
\newcommand{\aph}{\alpha_1}
\newcommand{\hang}[1]{\makebox[0cm]{#1}}
\newcommand{\hangs}[1]{\hang{\ #1}}
\let\oldtau\tau
\renewcommand{\tau}{\oldtau_1}
\newcommand{\taub}{\oldtau_B}
\newtheorem{proposition}{Proposition}[section]
\newtheorem{conjecture}[proposition]{Conjecture}
\newtheorem{lemma}[proposition]{Lemma}
\newtheorem{theorem}[proposition]{Theorem}
\newtheorem{corollary}[proposition]{Corollary}
\theoremstyle{definition}
\newtheorem{definition}[proposition]{Definition}
\theoremstyle{remark}
\newtheorem*{remark}{Remark}
\begin{document}
\begin{abstract}
  Erd\H os, Gallai, and Tuza posed the following problem: given an
  $n$-vertex graph $G$, let $\tau(G)$ denote the smallest size of a
  set of edges whose deletion makes $G$ triangle-free, and let
  $\aph(G)$ denote the largest size of a set of edges containing at
  most one edge from each triangle of $G$. Is it always the case that
  $\aph(G) + \tau(G) \leq n^2/4$?

  We have two main results. We first obtain the upper bound $\aph(G) +
  \tau(G) \leq 5n^2/16$, as a partial result towards the Erd\H
  os--Gallai--Tuza conjecture.  We also show that always $\aph(G) \leq
  n^2/2 - m$, where $m$ is the number of edges in $G$; this bound is
  sharp in several notable cases and suggests a strategy for proving
  the conjecture.
\end{abstract}
\maketitle
\section{Introduction}
Given an $n$-vertex graph $G$, say that a set $A \subset E(G)$ is
\emph{triangle-independent} if it contains at most one edge from each
triangle of $G$, and say that $X \subset E(G)$ is a \emph{triangle
  edge cover} if $G-X$ is triangle-free. Throughout this paper,
$\aph(G)$ denotes the maximum size of a triangle-independent set
of edges in $G$, while $\tau(G)$ denotes the minimum size of a triangle
edge cover in $G$.

Erd\H os~\cite{largebip} showed that every $n$-vertex graph $G$ has a
bipartite subgraph with at least $\sizeof{E(G)}/2$ edges, which implies that
$\tau(G) \leq \sizeof{E(G)}/2 \leq n^2/4$. Similarly, if $A$ is
triangle-independent, then the subgraph of $G$ with edge set $A$ is
clearly triangle-free; by Mantel's~Theorem, this implies that $\aph(G)
\leq n^2/4$.

Intuitively, $\aph(G)$ and $\tau(G)$ cannot both be large: if
$\tau(G)$ is close to $n^2/4$, then $\sizeof{E(G)}$ is close to
$n^2/2$, which makes it difficult to find a large triangle-independent
set of edges. Erd\H os, Gallai, and Tuza formalized this intuition
with the following conjecture.
\begin{conjecture}[Erd\H os--Gallai--Tuza \cite{EGT}]\label{coj:EGT}
  For every $n$-vertex graph $G$, $\aph(G) + \tau(G) \leq n^2/4$.
\end{conjecture}
The conjecture is sharp, if true: consider the graphs $K_{n}$ and
$K_{n/2, n/2}$, where $n$ is even. We have
\begin{align*}
  \aph(K_n) &= n/2, & \tau(K_{n}) &= {n \choose 2} - n^2/4; \\
  \aph(K_{n/2, n/2}) &= n^2/4, & \tau(K_{n/2,n/2}) &= 0.
\end{align*}
In both cases, $\aph(G) + \tau(G) = n^2/4$, but a different term
dominates in each case. As observed by Erd\H os, Gallai, and Tuza, the
difficulty of the conjecture lies in the variety of graphs for which
the conjecture is sharp: any proof of the conjecture would need to
account for both $K_n$ and $K_{n/2,n/2}$ without any waste.

Erd\H os, Gallai, and Tuza~\cite{EGT} considered the conjecture on
graphs for which every edge lies in a triangle, and proved that there
is a positive constant $c$ such that $\aph(G) + \tau(G) \leq
\sizeof{E(G)} - c\sizeof{E(G)}^{1/3}$ and $\aph(G) + \tau(G) \leq
\sizeof{E(G)} - c\sizeof{V(G)}^{1/2}$ for such graphs. Aside from the
original paper of Erd\H os, Gallai, and Tuza, no other work appears to
have been done on the conjecture.

In this paper, we present two partial results towards
Conjecture~\ref{coj:EGT}.

In Section~\ref{sec:indbip}, we extend some ideas of Erd\H os,
Faudree, Pach and Spencer~\cite{howbip} in order to obtain the bound
$\aph(G) + \tau(G) \leq 5n^2/16$. To our knowledge, this is the best
general bound on $\aph(G) + \tau(G)$.

In Section~\ref{sec:aphbound}, we obtain the bound $\aph(G) \leq n^2/2
- m$, where $m = \sizeof{E(G)}$, and characterize the graphs for which
equality holds. When $n$ is even, this bound is sharp for both $K_n$
and $K_{n/2,n/2}$, which makes it an encouraging step towards the
Erd\H os--Gallai--Tuza Conjecture.

\section{Induced Bipartite Subgraphs}\label{sec:indbip}
In this section, we will focus on the relationship between
triangle-free subgraphs of $G$ and bipartite subgraphs of $G$.  The
problem of finding a largest bipartite subgraph of a graph $G$ is
well-studied, and clearly any bipartite subgraph of $G$ is
triangle-free, so we can reasonably hope to apply some of the existing
literature to our current problem.

Erd\H os, Faudree, Pach, and Spencer~\cite{howbip} studied the problem
of finding a largest bipartite subgraph of a triangle-free graph,
using the observation that $uv$ is an edge of a triangle-free graph,
then $G[N(u) \cup N(v)]$ is an induced bipartite subgraph of
$G$. Here, we use a similar observation: if $A$ is a
triangle-independent set of edges in $G$, then for any $uv \in E(G)$,
the induced subgraph $G[N(u) \cup N(v)]$ is bipartite (even when the
edges in $E(G) - A$ are considered). The point is that $A$ is not
only triangle-free itself, but imposes restrictions on the triangles
in $G$.

We define some useful notation. For any graph $G$, let $\taub(G)$
denote the smallest size of an edge set $X$ such that $G-X$ is
bipartite, and let $b(G)$ denote the largest size of a vertex set $B$
such that $G[B]$ is bipartite.  Clearly, $\tau(G) \leq \taub(G)$, so
we seek bounds on $\aph(G) + \taub(G)$. When $A \subset E(G)$, we will
abuse notation slightly by identifying $A$ with the spanning subgraph
of $G$ having edge set $A$. This yields notation like $N_A(v)$,
referring to the neighborhood of a vertex $v$ in the spanning subgraph
of $G$ with edge set $A$.
\begin{lemma}\label{lem:edgebip}
  For any graph $G$, any triangle-independent set $A \subset E(G)$, and any edge $uv \in E(G)$,
  \[ d_A(u) + d_A(v) \leq b(G). \]
\end{lemma}
\begin{proof}
  Since $A$ is triangle-independent, the sets $N_A(u)$ and $N_A(v)$
  are independent and disjoint. Hence $G[N_A(u) \cup N_A(v)]$ is
  bipartite with $d_A(u) + d_A(v)$ vertices.
\end{proof}
\begin{lemma}\label{lem:bip-aph}
  For any graph $G$,
  \[ \aph(G) \leq \frac{nb(G)}{4\hangs.} \]
\end{lemma}
\begin{proof}
  Let $A$ be any triangle-independent subset of $E(G)$. Applying
  Lemma~\ref{lem:edgebip} to all edges in $A$ and summing together
  the resulting inequalities gives
  \[ \sum_{u \in V(G)}d_A(u)^2 = \sum_{uv \in A}[ d_A(u) + d_A(v) ] \leq \sizeof{A}b(G). \]
  By the Cauchy-Schwarz inequality, we have
  \[ \sum_{uv \in A}d_A(u)^2 \geq \frac{4\sizeof{A}^2}{n\hangs.} \]
  The desired inequality follows.
\end{proof}
\begin{lemma}\label{lem:bip-tau}
  For any graph $G$,
  \[ \taub(G) \leq \frac{n^2}{4} - \frac{b(G)^2}{4\hangs.} \]
\end{lemma}
\begin{proof}
  This is essentially the $\delta = 0$ case of Proposition~2.5 of
  \cite{howbip}.  We sketch a probabilistic proof here. Let $B$ be the
  vertex set of a largest bipartite induced subgraph of $G$. If we
  randomly place the vertices of $V(G) - B$ into the partite sets of
  $B$, the expected number of deleted edges is $(E(G) - E(G[B]))/2$;
  hence $G$ can be made bipartite by deleting at most this many edges.
  Hence
  \[ \taub(G) \leq \frac{1}{2}\sizeof{E(G) - E(G[B])} \leq \frac{1}{2}\left[{n \choose 2} - {b(G) \choose 2}\right] \leq \frac{n^2}{4} - \frac{b(G)^2}{4\hangs.}\qedhere\]
\end{proof}
\begin{corollary}\label{cor:nearperf}
  For any graph $G$,
  \[ \aph(G) + \taub(G) \leq \frac{5n^2}{16\hangs.} \]  
\end{corollary}
\begin{proof}
  From Lemma~\ref{lem:bip-aph} and Corollary~\ref{lem:bip-tau} we immediately have
  \[ \aph(G) + \taub(G) \leq \frac{n^2}{4} + \frac{nb(G)}{4} - \frac{b(G)^2}{4\hangs.} \]
  By freshman calculus, maximizing the right-hand function over $b(G)$ gives
  \[ \aph(G) + \taub(G) \leq \frac{5n^2}{16\hangs,} \]
  as desired.
\end{proof}
\section{Bounding $\aph(G)$}\label{sec:aphbound}
In this section, we will obtain the bound $\aph(G) \leq \lfrac{n^2}{2} - m$.
We first need one quick lemma.
\begin{lemma}\label{lem:vertmax}
  Let $G$ be an $n$-vertex graph, and let $A \subset E(G)$ be
  triangle-independent.  For every edge $uv \in A$, we have $d_A(u)
  \leq n - d_G(v)$.
\end{lemma}
\begin{proof}
  The set $A$ cannot contain any edge $uw$ where $w \in N_G(v)$, since
  then $A$ would contain two edges of the triangle $uvw$. Hence $N_A(u) \subset V(G) - N_G(w)$.
\end{proof}
\begin{theorem}\label{thm:match}
  For an $n$-vertex graph $G$ with $m$ edges,
  \[ \aph(G) \leq \frac{n^2}{2} - m. \]
  Equality holds if and only if there exist $r_1, \ldots, r_t \geq 1$
  such that $G \iso K_{r_1,r_1,\ldots,r_t,r_t}$.
\end{theorem}
\begin{proof}
  Let $A \subset E(G)$ be triangle-independent, and let $M$ be a
  maximal matching in $A$.  We study the degree sum $\sum_{v \in
    V(G)}d_A(v)$ by splitting it into the sum $\sum_{v \in V(M)}d_A(v)
  + \sum_{v \notin V(M)}d_A(v)$.

  For each $v$ covered by $M$, let $v'$ be its mate in $M$.  Applying
  Lemma~\ref{lem:vertmax} to both endpoints of each edge in $M$, we
  obtain the bound
  \[
    \sum_{v \in V(M)}d_A(v) \leq \sum_{v \in V(M)}[n - d_G(v')] = \sum_{v \in V(M)}[n - d_G(v)].
    \]
  To bound $\sum_{v \notin V(M)}d_A(v)$, we first observe that the vertices not
  covered by $M$ form an independent set in $A$, since any edge
  joining such vertices could be added to obtain a larger matching.

  Now let $v$ be any vertex not covered by $M$.  For each edge $ww'
  \in M$, if $vw \in A$ then $vw' \notin E(G)$, since otherwise $A$
  contains two edges of the triangle $vww'$. Hence $d_A(v) \leq n - 1 -
  d_G(v)$, since each $A$-edge $vw$ is witnessed by a non-$G$-edge $vw'$ where $w' \neq v$.
  Summing this inequality over all uncovered $v$ yields
  \begin{equation}
    \label{eq:cat}
    \sum_{v \notin V(M)}d_A(v) \leq \sum_{v \notin V(M)}[n - 1 - d_G(v)] \leq \sum_{v \notin V(M)}[n - d_G(v)].
  \end{equation}
  Combining this with the bound on $\sum_{v \in V(M)}d_A(V)$ and applying the degree-sum formula for $G$ yields
  \[ \sum_{v \in V(G)}d_A(v) \leq \sum_{v \in V(G)}[n - d_G(v)] = n^2 - 2m. \]
  Applying the degree-sum formula for $A$ completes the proof of the first claim.
  \medskip

  Now we characterize the graphs for which equality holds.  If
  $\sizeof{A} = \lfrac{n^2}{2} - m$, then every maximal matching in $A$
  is perfect, since $M$ was an arbitrary maximal matching in $A$ and
  equality can only hold in \eqref{eq:cat} if the sum is empty.

  Let $P_4$ denote the path on four vertices. We claim that if $A$
  contains an induced subgraph isomorphic to $P_4$, then $A$ contains a
  nonperfect maximal matching. For let $v_1,\ldots,v_4$ be the
  vertices of an induced copy of $P_4$ in $A$, written in order, and
  let $M$ be any maximal matching containing the edges $v_1v_2$ and
  $v_3v_4$; we may assume that $M$ is a perfect matching. Now $M -
  \{v_1v_2,v_3v_4\} + \{v_2v_3\}$ is a nonperfect maximal matching,
  since $v_1v_4 \notin A$.

  Thus, if equality holds in the bound, then $A$ is a triangle-free
  graph with a perfect matching and no induced $P_4$. This implies
  that every component of $A$ is a balanced complete bipartite graph.

  Next we claim that if $u$ and $v$ are vertices in different
  components of $A$, then $uv \in E(G)$. Suppose $uv \notin E(G)$, and
  let $G' = G + uv$.  Now $A$ still contains at most one edge from
  any triangle of $G'$: if not, then $A$ contains two edges of $uvz$,
  for some $z \in V(G')$.  Since $uv \notin A$, this implies that $uz
  \in A$ and $vz \in A$, contradicting the hypothesis that $u$ and $v$
  are in different components of $A$. It follows that $\sizeof{A} \leq
  \lfrac{n^2}{2} - \sizeof{E(G')} < \lfrac{n^2}{2} - m$, contradicting
  the assumption that $\sizeof{A} = \lfrac{n^2}{2} - m$.

  We have shown that the vertices of $G$ can be covered by
  vertex-disjoint complete bipartite graphs, and that if $u$ and $v$
  are covered by different graphs, then $uv \in E(G)$. This implies
  that $G$ is a complete multipartite graph of the form
  $K_{r_1,r_1,\ldots,r_t,r_t}$, as desired.
\end{proof}
Theorem~\ref{thm:match} suggests a possible approach to proving
Conjecture~\ref{coj:EGT}. Observe that we have an obvious lower bound
of $\tau(G) \geq m - \lfrac{n^2}{4}$, since Mantel's~Theorem says that
any triangle-free subgraph of $G$ has at most $n^2/4$ edges.

Let $k$ and $\ell$ be defined by
\begin{align*}
  \aph(G) &= \frac{n^2}{2} - m - k, \\
  \tau(G) &= m - \frac{n^2}{4} + \ell.
\end{align*}
The bounds $\aph(G) \leq n^2/2 - m$ and $\tau(G) \geq m - n^2/4$ imply
that $k, \ell \geq 0$. Now $\aph(G) + \tau(G) = \lfrac{n^2}{4} + (\ell
- k)$, so for Conjecture~\ref{coj:EGT} to hold for $G$, we
need $\ell \leq k$. As such, Conjecture~\ref{coj:EGT} is
equivalent to the following claim, expressed in terms of these bounds:
\begin{conjecture}\label{coj:EGT2}
  For all graphs $G$ and all $k \geq 0$, if $\aph(G) \geq \lfrac{n^2}{2} - m - k$,
  then $\tau(G) \leq m - \lfrac{n^2}{4} + k$.
\end{conjecture}
The characterization of graphs with $\aph(G) = \lfrac{n^2}{2} - m$
is the $k=0$ case of Conjecture~\ref{coj:EGT2}.
\bibliographystyle{amsplain}\bibliography{sumbib}
\end{document}
