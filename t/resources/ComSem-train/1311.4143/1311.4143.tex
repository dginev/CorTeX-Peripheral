\documentclass[a4paper,12pt]{article}
\setlength{\textwidth}{14.4cm} \setlength{\textheight}{22cm}

\title{The Weierstrass semigroups on double covers of genus two curves}
\author{
Takeshi Harui\thanks{E-mail: takeshi@cwo.zaq.ne.jp, kt13459@ns.kogakuin.ac.jp}\\
{\small Academic Support Center, Kogakuin University}\\
{\small  Hachioji, 192-0015, Japan}\\
Jiryo Komeda\thanks{E-mail: komeda@gen.kanagawa-it.ac.jp}\\
{\small Department of Mathematics, Center for Basic Education and Integrated Learning}\\
{\small Kanagawa Institute of Technology, Atsugi, 243-0292, Japan}\\
and\\
Akira Ohbuchi\thanks{E-mail: ohbuchi@tokushima-u.ac.jp}\\
{\small Department of Mathematics, Faculty of Integrated Arts and Sciences}\\
{\small Tokushima University, Tokushima, 770-8502, Japan}\\
}

\date{}

\usepackage{amssymb}
\usepackage[leqno]{amsmath}
\usepackage{latexsym}
%\usepackage{leqno}
\usepackage{theorem}
\usepackage{enumerate}
\pagestyle{myheadings} \markright{ } \setcounter{section}{0}
\setcounter{secnumdepth}{3}

\newtheorem{mtheorem}{Theorem}
\newtheorem{theorem}{Theorem}[section]
\newtheorem{corollary}[theorem]{Corollary}
\newtheorem{proposition}[theorem]{Proposition}
\newtheorem{lemma}[theorem]{Lemma}

\newtheorem{remark}[theorem]{Remark}

\newcounter{examplec}[section]

\newcommand{\example}
{\vskip 10pt \noindent {\bf Example
\thesection.\,\stepcounter{examplec}\theexamplec\
\hspace*{1.5mm}}}

\newcommand{\proof}{\noindent \mbox{\em Proof.\hspace*{2mm}}}
\newcommand{\qed}{\hfill $  \Box $}


\renewcommand{\labelenumi}{(\theenumi)}  %item%enumerate

\newcommand{\Spe}{\mbox{Spec }}
%\newcommand{\Tdim}{\mbox{Tdim }}
\newcommand{\tH}{\tilde{H}}
\newcommand{\tC}{\tilde{C}}
\newcommand{\tP}{\tilde{P}}
\newcommand{\tg}{\tilde{g}}
\newcommand{\tih}{\tilde{h}}
%\newcommand{\qed}{\hfill $  \Box $}
\newcommand{\la}{\langle}
\newcommand{\ra}{\rangle}
\newcommand{\pr}{\vskip3mm
\noindent
{\it Proof.} }
\newcommand{\gaps}{\mathbb{N}_0\backslash}
\newcommand{\smax}{s_{{\rm max}}}
\newcommand{\bu}{\bullet}
\newcommand{\dis}{\displaystyle}
\newcommand{\fn}{\frac{n-1}{2}}
\newcommand{\NI}{\mathbb{N}_0}
\newcommand{\rn}{\romannumeral}
\newcommand{\DC}{\mbox{of double covering type}}
\newcommand{\dc}{\mbox{of double covering type}}
\newcommand{\com}{\mbox{, }}
\Roman{equation}
\begin{document}
\maketitle
\renewcommand{\thefootnote}{\fnsymbol{footnote}}
\footnotetext{
The second author is partially supported by
Grant-in-Aid for Scientific Research (24540057), Japan Society for the Promotion Science.
The third author is partially supported by
Grant-in-Aid for Scientific Research (24540042), Japan Society for the Promotion Science.
 }
%\renewcommand{\labelenumi}{\thesection.\theenumi}
\renewcommand{\abstractname}{}
\begin{abstract}\vskip-2mm
We show that three numerical semigroups $\la 5,6,7,8 \ra$, $\la 3,7,8 \ra$ and $\la 3,5 \ra$ are of double covering type, i.e., the Weierstrass semigroups of ramification points on double covers of curves.
Combining the result with~\cite{oli-pim} and~\cite{kom2} we can determine the Weierstrass semigroups of the ramification points on double covers of genus two curves.

\vspace{2mm} \noindent 
{\bf 2010 Mathematics Subject
Classification:} 14H55, 14H45, 20M14 \\
{\bf Key words:} Numerical semigroup, Weierstrass semigroup, Double cover of a curve, Curve of genus two
\end{abstract}
%\newpage
\section{Introduction}
\label{intro}
Let $C$ be a complete nonsingular irreducible curve over an algebraically closed field $k$ of characteristic 0, which is called a {\it
curve} in this paper.
For a point $P$ of $C$, we set
$$H(P)=\{\alpha\in \mathbb{N}_{0}|\mbox{ there exists a rational function } f \mbox{ on }C\mbox{ with } (f)_{\infty}=\alpha P\},$$
which is called the {\it Weierstrass semigroup of $P$} where $\mathbb{N}_{0}$ denotes the additive monoid of non-negative integers.
A submonoid $H$ of $\mathbb{N}_0$ is called a {\it numerical semigroup} if its complement $\mathbb{N}_0\backslash H$ is a finite set.
The cardinality of $\mathbb{N}_0\backslash H$ is called the {\it genus} of $H$, which is denoted by $g(H)$.
It is known that the Weierstrass semigroup of a point on a curve of genus $g$ is a numerical semigroup of genus $g$.
For a numerical semigroup $\tH$ we denote by $d_2(\tH)$ the set of consisting of the elements  $\tih/2$ with even $\tih\in \tH$, which becomes a numerical semigroup.
A numerical semigroup $\tH$ is said to be {\it of double covering type} if there exists a double covering $\pi:{\tilde C}\longrightarrow C$ of a curve with a ramification point $\tP$ over $P$ satisfying $H(\tP)=\tH$.
In this case we have $d_2(H(\tP))=H(P)$.

We are interested in numerical semigroups of double covering type.
Let $\tH$ be a numerical semigroup of genus $\tg$ with $d_2(\tH)=\NI$ whose genus is $0$.
Then the semigroup $\tH$ is  $\la 2,2\tg+1\ra$, where for any positive integers $a_{1},a_{2},\ldots,a_{n}$ we denote by $\langle a_{1},a_{2},\ldots,a_{n}\rangle$ the additive monoid generated by $a_{1},a_{2},\ldots,a_{n}$.
In this case $\tH$ is the Weierstrass semigroup of a ramification point $\tP$ on a double cover of the projective line which is of genus $\tg$.
Hence, $\tH$ is \DC.

Let $\tH$ be a numerical semigroup of genus $\tg$ with $d_2(\tH)=\la 2,3 \ra$ which is the only one numerical semigroup of genus $1$.
Then the semigroup $\tH$ is either $\la 3,4,5\ra$ or $\la 3,4\ra$ or $\la 4,5,6,7\ra$ or $\langle 4,6,2\tg-3\rangle$ with $\tg\geqq 4$ or $\langle 4,6,2\tg-1,2\tg+1\rangle$ with $\tg\geqq 4$.
We can show that there is a double covering of an elliptic curve with a ramification point whose Weierstrass semigroup is any semigroup in the above ones (for example, see~\cite{kom1},~\cite{kom2}).

Oliveira and Pimentel~\cite{oli-pim} studied the semigroup $\tH=\langle 6,8,10,n\rangle$ with an odd number $n\geqq 11$.
They showed that the semigroup $\tH$ is \DC.
In this case we have $d_2(\tH)=\la 3,4,5 \ra$, which is of genus $2$.
Moreover, in~\cite{kom2} we proved that any numerical semigroup $\tH$ with $d_2(\tH)=\la 3,4,5\ra$ except $\la 5,6,7,8 \ra$, $\la 3,7,8 \ra$, $\la 3,5 \ra$ and $\la 3,5,7 \ra$ is \DC.
In view of the fact that $g(\la 3,5,7 \ra)=3<2\cdot 2$ the semigroup $\la 3,5,7 \ra$ is not \DC.
There is another numerical semigroup of genus $2$, which is $\la 2,5 \ra$.
Using the result of Main Theorem in~\cite{kom-ohb} every numerical semigroup $\tH$ with $d_2(\tH)=\la 2,5 \ra$ is \DC.
In this paper we will study the remaining three numerical semigroups.
Namely we prove the following:
\vskip3mm
\noindent
{\bf Theorem 1} {\it The three numerical semigroups $\la 5,6,7,8 \ra$, $\la 3,7,8 \ra$ and $\la 3,5 \ra$ are of double covering type.}
\vskip3mm
Combining this theorem with the results in~\cite{oli-pim} and~\cite{kom2}, we have the following conclusion:
\vskip3mm
\noindent
{\bf Theorem 2} {\it Let $\tH$ be a numerical semigroup with $g(d_2(\tH))=2$.
If $\tH\not=\la 3,5,7\ra$, then it is of double covering type.} 
%
%
\section{The proof of Theorem}
%
To prove that the three numerical semigroups are \DC  \ we use the following remark which is stated in Theorem 2.2 of~\cite{kom3}.
\vskip3mm
\noindent
{\bf Remark 1. }Let $\tH$ be a numerical semigroup.
We set
$$n=\min\{\tih\in \tH\mid \tih\mbox{ is odd}\}\mbox{ and }g(\tH)=2g(d_2(\tH))+\frac{n-1}{2}-r$$
with some non-negative integer $r$.
Assume that $H=d_2(\tH)$ is Weierstrass.
Take a pointed curve $(C,P)$ with $H(P)=H$.
Let $Q_1,\ldots,Q_r$ be points of $C$ different from $P$ with $h^0(Q_1+\cdots+Q_r)=1$.
Moreover, assume that $\tH$ has an expression
$$\tH=2H+\la n,n+2l_1,\ldots,n+2l_s\ra$$
of generators with positive integers $l_1,\ldots,l_s$ such that
$$h^0(l_iP+Q_1+\cdots+Q_r)=h^0((l_i-1)P+Q_1+\cdots+Q_r)+1$$
for all $i$.
If the divisor $nP-2Q_1-\cdots-2Q_r$ is linearly equivalent to some reduced divisor not containing $P$, then there is a double covering $\pi:\tC\longrightarrow C$ with a ramification point $\tP$ over $P$ satisfying $H(\tP)=\tH$, hence $\tH$ is $\DC$.
\vskip3mm
\noindent
By seeing the proof of Theorem 2.2 in~\cite{kom3} we may replace the assumption in Theorem 2.2 in~\cite{kom3} that the complete linear system $|nP-2Q_1-\cdots-2Q_r|$ is base point free by the above assumption that the divisor $nP-2Q_1-\cdots-2Q_r$ is linearly equivalent to some reduced divisor not containing $P$.
\vskip3mm
\noindent
{\it Case 1.} Let $\tH=\la 5,6,7,8\ra$.
Then we have $H=d_2(\tH)=\la 3,4,5\ra$ and $\dis g(\tH)=5=2\cdot 2+\frac{5-1}{2}-1$.
Moreover, we have $\tH=2H+\la 5,5+2\cdot 1\ra$.
Let $C$ be a curve of genus $2$ and $\iota$ the hyperelliptic involution on $C$.
Let us take a point $P$ of $C$ with $H(P)=\la 3,4,5\ra$ and $3(P-\iota(P))\not\sim 0$.
Then we get $h^0(P+\iota(P))=2=h^0(\iota(P))+1$.
Moreover, we have $R\not=P$ if the complete linear system $|5P-2\iota(P)|$ has a base point $R$.
Indeed, we assume that $R=P$.
Then we have
$$h^0(5P-2\iota(P)-P)=h^0(5P-2\iota(P))=3+1-2=2,$$
which implies that
$$4P-2\iota(P)\sim g_2^1\sim P+\iota(P).$$
Hence, we get $3(P-\iota(P))\sim 0$.
This is a contradiction.

We assume that $|5P-2\iota(P)|$ has a base point $R$.
Then we get $5P-2\iota(P)\sim R+E$, where $E$ is an effective divisor of degree $2$ with projective dimension $1$.
In this case the complete linear system $|E|$ is base point free. Therefore, the divisor $5P-2\iota(P)$ is linearly equivalent to some reduced divisor not containing $P$.
If $|5P-2\iota(P)|$ is base point free, then the divisor $5P-2\iota(P)$ satisfies the above condition.
By Remark 1 the semigroup $\tH=\la 5,6,7,8\ra$ is \DC.
\vskip3mm
\noindent
{\it Case 2.} Let $\tH=\la 3,7,8\ra$.
Then we have $H=d_2(\tH)=\la 3,4,5\ra$ and $\dis g(\tH)=4=2\cdot 2+\frac{3-1}{2}-1$.
Moreover, we have $\tH=2H+\la 3,3+2\cdot 2\ra$.
Let $C$ be a curve of genus $2$ and $\iota$ the hyperelliptic involution on $C$.
We take a point $P$ of $C$ with $H(P)=\la 3,4,5\ra$.
Let $\varphi:C\longrightarrow \mathbb{P}^1$ be a covering of degree $3$ corresponding to the complete linear system $|3P|$.
We may take the pointed curve $(C,P)$ such that $\varphi$ has a simple ramification point $Q$.
Then there is another simple ramification point
%$Q'$ 
of $\varphi$ by Riemann-Hurwitz formula.
Hence, we may assume that $\iota P\not=Q$, which implies that $P+Q\not\sim g_2^1$.
Thus, we get $h^0(2P+Q)=2=h^0(P+Q)+1$.
Let $R$ be the point satisfying $2Q+R\sim 3P$.
Then we have $R\not=P$ and $3P-2Q\sim R$.
By Remark 1 the semigroup $\tH=\la 3,7,8\ra$ is \DC.
\vskip3mm
\noindent
{\it Case 3.} Let $\tH=\la 3,5\ra$.
Then we have $H=d_2(\tH)=\la 3,4,5\ra$ and $\dis g(\tH)=4=2\cdot 2+\frac{3-1}{2}-1$.
Moreover, we have $\tH=2H+\la 3,3+2\cdot 1\ra$.
Let $C$ be a curve whose function field is $k(x,y)$ with an equation $y^3=(x-c_1)(x-c_2)(x-c_3)^2$, where $c_1$, $c_2$ and $c_3$ are
distinct elements of $k$.
Let $\pi:C\longrightarrow \mathbb{P}^1$ be the morphism corresponding to the inclusion $k(x)\subset k(x,y)$.
Then $C$ is of genus $2$.
Let $P=P_1$, $P_2$, $P_3$ and $P_4$ be the ramification points of $\pi$.
Since $\pi$ is a cyclic covering, it induces an automorphism $\sigma$ of $C$ with $C/\la \sigma \ra\cong \mathbb{P}^1$.
Let $\iota$ be the hyperelliptic involution on $C$.
Then we have $\sigma\circ\iota=\iota\circ\sigma$.
Indeed, we have
$$(\sigma\circ\iota\circ\sigma^{-1})\circ(\sigma\circ\iota\circ\sigma^{-1})=\sigma\circ\iota\circ\iota\circ\sigma^{-1}=\sigma\circ\sigma^{-1}=id.$$
Hence, the automorphism $\sigma\circ\iota\circ\sigma^{-1}$ is an involution.
Moreover, we have a bijective correspondence between the sets Fix$(\iota)$ and Fix$(\sigma\circ\iota\circ\sigma^{-1})$
sending $Q$ to $\sigma(Q)$, where Fix$(\iota)$ and Fix$(\sigma\circ\iota\circ\sigma^{-1})$ are the sets of the fixed points
by $\iota$ and $\sigma\circ\iota\circ\sigma^{-1}$ respectively.
Hence, $\sigma\circ\iota\circ\sigma^{-1}$ is also the hyperelliptic involution.
Thus, we have $\sigma\circ\iota\circ\sigma^{-1}=\iota$.
Since $\sigma(\iota(P))=\iota(\sigma(P))=\iota(P)$, the point $\iota(P)$ is a fixed point of $\sigma$.
Hence, we may assume that $\iota(P)=P_2$.
Then we obtain $h^0(P+P_2)=2=h^0(P_2)+1$.
Moreover, we have
$$3P-2P_2\sim 3P_2-2P_2=P_2.$$
By Remark 1 the semigroup $\la 3,5\ra$ is \DC.
\qed
%
\begin{thebibliography}{00}
%
\bibitem{kom1} Komeda, J.: On Weierstrass points whose first non-gaps are four, J. reine angew. Math. \textbf{341}, 68--86 (1983).
%
\bibitem{kom2} Komeda, J.: A numerical semigroup from which the semigroup gained by dividing by two is
either $\mathbb{N}_0$ or a $2$-semigroup or $\langle 3,4,5\rangle$, Research Reports of Kanagawa Institute of
Technology \textbf{B-33}, 37--42 (2009).
%
\bibitem{kom3} Komeda, J.: On Weierstrass semigroups of double coverings of genus three curves, Semigroup Forum {\bf 83}, 479-488 (2011).
%
%
%\bibitem{kom-ohb1} Komeda, J., Ohbuchi, A.: Weierstrass points with first non-gap four on a double covering of
%a hyperelliptic curve, Serdica Math. J. \textbf{30}, 43--54 (2004). 
%
%\bibitem{kom-ohb} Komeda, J., Ohbuchi, A.: On double coverings of a pointed non-singular curve with any Weierstrass semigroup, Tsukuba J. Math. \textbf{31}, 205--215 (2007).
%
\bibitem{kom-ohb} Komeda, J., Ohbuchi, A.: Weierstrass points with first non-gap four on a double covering of a hyperelliptic
curve II, Serdica Math. J. \textbf{34}, 771--782 (2008). 
%
\bibitem{oli-pim} Oliveira, G., Pimentel, F. L. R.: On Weierstrass semigroups of double covering of genus two curves,
Semigroup Forum \textbf{77}, 152--162  (2008). 
%
%\bibitem{ol-to-vi} Oliveira, G., Torres, F., Villanueva, J.: On the weight of numerical semigroups, J. Pure Appl. Algebra \textbf{214}, 1955--1961 (2010).
%
%\bibitem{tor} Torres, F.: Weierstrass points and double coverings of curves with application: Symmetric numerical semigroups which cannot be realized as Weierstrass semigroups. {\it Manuscripta Math.} \textbf{83}:39--58 (1994).

\end{thebibliography} 


\end{document}





\bibitem{buch} R.O. Buchweitz, {\it On Zariski's criterion for equisingularity and non-smoothable monomial curves}, Preprint 113, University of Hannover, 1980.
%
\bibitem{hurw} A. Hurwitz, {\it \"{U}ber algebraischer Gebilde mit eindeutigen Transformationen in sich}, Math. Ann. 41 (1893) 403-442.
%
\bibitem{komp} J. Komeda,  {\it On the existence of Weierstrass gap sequences on curves of genus $\leq 8$}, J. Pure Appl. Alg. {\bf 97} (1994) 51-71.
%
\bibitem{koms} J. Komeda,  {\it On Weierstrass semigroups of double coverings of genus three curves}, Semigroup Forum {\bf 83} (2011) 479-488.
%
\bibitem{komc} J. Komeda,  {\it Double coverings of curves and non-Weierstrass semigroup}, Communications in Algebra {\bf 41} (2013) 312-324.
%
%
\bibitem{komk} J. Komeda,  {\it On quasi-symmetric numerical semigroups}, Research reports of Kanagawa Institute of Technology  {\bf B-35} (2011) 17-21.
%
\bibitem{koma} J. Komeda,  {\it Weierstrass semigroups on double covers of genus four curves}, arXiv:1310.1543 (2013). 
%
\bibitem{kom-ohb} J. Komeda and A. Ohbuchi,  {\it Existence of the non-primitive Weierstrass gap sequences on curves of genus $8$}, Bull. Braz. Math. Soc. {\bf 39} (2008) 109-121.
%
\bibitem{kom-ohb2} J. Komeda and A. Ohbuchi,  {\it On double coverings of a pointed no-singular curve with any Weierstrass semigroup}, Tsukuba J. Math. Soc. {\bf 31} (2007) 205-215.
%
\bibitem{kom-ohb3} J. Komeda and A. Ohbuchi,  {\it Weierstrass points with first non-gap four on a double covering of a hyperelliptic curve}, Serdica Math. J. {\bf 30} (2004) 43-54.
%
\bibitem{oli} G. Oliveira, {\it Weierstrass semigroups and the canonical ideal of non-trigonal curves,} Manuscripta Math. {\bf 71} (1991) 431-450.
%
\bibitem{tor} F. Torres, {\it Weierstrass points and double coverings of curves with application: Symmetric numerical semigroups which cannot be realized as Weierstrass semigroups,} Manuscripta Math. {\bf 83} (1994) 39-58.

















